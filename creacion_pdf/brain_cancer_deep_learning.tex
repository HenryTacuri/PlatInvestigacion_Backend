% This is samplepaper.tex, a sample chapter demonstrating the
% LLNCS macro package for Springer Computer Science proceedings;
% Version 2.21 of 2022/01/12
%
\documentclass[runningheads]{llncs}
\titlerunning{Universidad Politecnica Salesiana}

%
\usepackage[T1]{fontenc}
% T1 fonts will be used to generate the final print and online PDFs,
% so please use T1 fonts in your manuscript whenever possible.
% Other font encondings may result in incorrect characters.
%º
% Used for displaying a sample figure. If possible, figure files should
% be included in EPS format.
%
% If you use the hyperref package, please uncomment the following two lines
% to display URLs in blue roman font according to Springer's eBook style:
%\usepackage{color}
%\renewcommand\UrlFont{\color{blue}\rmfamily}
%
\usepackage{hyperref}
\usepackage{float}
\usepackage{multirow}
\usepackage{cite}
\usepackage{breqn}
\usepackage{amsmath,amssymb,amsfonts}
\usepackage{textcomp}
\usepackage{subfigure}
\usepackage{soul}
\usepackage[utf8]{inputenc}
\usepackage{tabularx} % Importa el paquete
\usepackage{amsmath} % assumes amsmath package installed
\usepackage{amssymb} 
\usepackage{array}
\usepackage{algpseudocode}
\usepackage{blindtext}
\usepackage{color}
\usepackage{algorithm}
\usepackage{epstopdf}
\usepackage{placeins}
\usepackage{wrapfig}
\usepackage{graphicx} % Necesario para \includegraphics
\usepackage{enumitem}

\restylefloat{algorithm}
\newcommand{\INDSTATE}[1][1]{\STATE\hspace{#1\algorithmicindent}}
\newcounter{mytempeqncnt}
\def\BibTeX{{\rm B\kern-.05em{\sc i\kern-.025em b}\kern-.08em
		T\kern-.1667em\lower.7ex\hbox{E}\kern-.125emX}}
	
\begin{document}
%
\title{Survey of brain cancer deep learning}
%
%\titlerunning{Abbreviated paper title}
% If the paper title is too long for the running head, you can set
% an abbreviated paper title here
%
\author{Remigio Hurtado\inst{1}}
%
% First names are abbreviated in the running head.
% If there are more than two authors, 'et al.' is used.
%
\institute{Universidad Polit\'ecnica Salesiana, Calle Vieja 12-30 y Elia Liut, Cuenca, Ecuador. \email{rhurtadoo@ups.edu.ec}\\
\url{ups.edu.ec}}
%
\maketitle    
%
\begin{abstract}
The application of deep learning techniques in brain cancer detection, particularly through microwave imaging, has significantly advanced the field, addressing crucial diagnostic challenges associated with tumor identification. This study introduces the Fine	extendash{}tuned Feature Extracted Deep Transfer Learning (FT	extendash{}FEDTL) model, which utilizes the InceptionV3 architecture to enhance the classification accuracy of brain tumors, achieving a remarkable accuracy rate of 99.65 percent across a dataset of 4200 microwave	extendash{}based images. The model categorizes tumors into six classes, demonstrating exceptional performance metrics, including high precision, recall, and specificity. By automating the classification process, the FT	extendash{}FEDTL model not only improves diagnostic efficiency but also alleviates the cognitive burden on radiologists, facilitating a more effective response to complex cases.

The research highlights the importance of early detection in improving patient outcomes, as the incidence rates of brain cancer continue to rise. Furthermore, comparative analyses reveal that the FT	extendash{}FEDTL model outperforms traditional machine learning methods, particularly in managing class imbalances and extracting relevant features from complex datasets. The implications of this work extend beyond classification, underscoring the transformative potential of deep learning in medical imaging and its role in integrating artificial intelligence solutions into clinical practice.

Additionally, the study emphasizes the need for ongoing research into deep learning frameworks and their applications in diverse datasets to enhance diagnostic capabilities for various cancers. By reinforcing the significance of methodological advancements and the integration of multi	extendash{}omics data with histopathological analysis, this research paves the way for more personalized treatment approaches and improved patient outcomes in brain cancer management.
\end{abstract}

\begin{keywords}
brain cancer, deep learning, microwave imaging, glioblastoma, tumor classification
\end{keywords}

\section{Introduction}
Recent advancements in deep learning have brought significant improvements to the detection and classification of brain cancer, particularly through innovative techniques such as microwave brain imaging (MBI). MBI technologies leverage the unique electromagnetic properties of brain tissues for early tumor detection, enabling more accurate and efficient diagnostics that capitalize on advanced algorithms. One notable development is the Fine	extendash{}tuned Feature Extracted Deep Transfer Learning (FT	extendash{}FEDTL) model, which utilizes the InceptionV3 architecture to enhance the classification capabilities of brain tumor images. However, the complexity and variability of tumor shapes pose considerable challenges for manual identification and classification by radiologists. Consequently, deep learning solutions like FT	extendash{}FEDTL automate the classification process, achieving impressive accuracies of up to 99.65%. Despite these advancements, the field still grapples with issues such as the interpretability of deep learning models and the need for high	extendash{}quality annotated datasets, which remain under	extendash{}explored in current literature.

To address these challenges, seminal works have emerged, including the FT	extendash{}FEDTL model, which illustrates the transformative potential of deep learning in clinical settings, emphasizing its integration into computer	extendash{}aided diagnosis systems. The systematic review of machine learning techniques in glioblastoma (GBM) highlights the ongoing need for improved methodologies and the integration of multi	extendash{}omics data with histopathological analysis for enhanced diagnostic accuracy and personalized treatment approaches. This survey aims to consolidate existing knowledge by systematically analyzing relevant literature through advanced natural language processing and machine learning methodologies, particularly utilizing Latent Dirichlet Allocation (LDA) for topic identification. By generating comprehensive knowledge representations and visualizations, this work intends to provide a user	extendash{}friendly report that synthesizes critical findings and underscores the importance of deep learning in advancing brain cancer detection and treatment strategies.

The key contributions of our study can be summarized as follows: 

\begin{itemize}
    \item \textbf{Exhaustive Review of the State of the Art}: Detailed analysis and synthesis of the most relevant and recent contributions in the existing literature.
    \item \textbf{Structured Methodology Based on CRISP	extendash{}DM}: Application of a standard methodology to guide the data mining process, including phases from understanding the problem to generating knowledge representations.
    \item \textbf{Advanced Machine Learning Models}: Use of Latent Dirichlet Allocation (LDA) to identify and categorize latent topics and development of a recommendation system based on the document	extendash{}topic matrix, in order to identify the most relevant themes and the most important associated documents.
    \item \textbf{Synthetized Knowledge Generation}: Creation of textual summaries and visualization of relationships between topics and documents to improve understanding.
    \item \textbf{Comprehensive and Understandable Reports}: Integration of representations in detailed reports that combine text and figures to provide a clear and complete view of the analysis.
\end{itemize}

In the remainder of the document, the following is presented: The development methodology of this study, then, the evaluation in the field, the topics: Brain Tumor Classification, SLCA UNet Segmentation, Deep Learning Glioblastoma,  subsequently, the trends in brain cancer deep learning, related works and original contributions, and finally, the conclusions.
\section{Methodology}

Our methodology is designed to ensure a comprehensive and systematic analysis of scientific literature. The methodology used in this study builds on our previous work, including a methodology that leverages machine learning and natural language processing techniques \cite{Hurtado2023}, and a novel method for predicting the importance of scientific articles on topics of interest using natural language processing and recurrent neural networks \cite{Lopez2024}. The process is structured into three phases: data preparation, topic modeling, and the generation and integration of knowledge representations. Each phase is essential for transforming raw text data into meaningful insights, and the detailed parameters and algorithm are explained below. The table \ref{tabDescription} describes the parameters for understanding the overall process and algorithm. The high-level process is presented in Fig. \ref{fig:Methodology}, and the detailed algorithm is outlined in Table \ref{tab:Algorithm}.\\ 

In the data preparation phase, we focus on extracting and cleaning text from scientific documents using natural language processing (NLP) techniques. This phase involves several steps to ensure that the text data is ready for analysis. First, text is extracted from the documents, and non-alphabetic characters that do not add value to the analysis are removed. Next, the text is converted to lowercase, and stopwords (common words that do not contribute much meaning) are removed. We then apply lemmatization, which transforms words to their base form (e.g., "running" becomes "run"). Each document is tokenized (split into individual words or terms), and n-grams (combinations of words) are identified to find common terms. We generate a unified set of common terms, denoted as $TE$, which includes both the terms extracted from the documents and basic terms relevant to any field of study, such as [Fundamentals, Evaluation of Solutions, Trends]. Finally, each document is vectorized with respect to $TE$, resulting in the Document-Term Matrix (DTM).\\

The DTM is a crucial component for topic modeling. It is a matrix where the rows represent the documents in the corpus, and the columns represent the terms (words or n-grams) extracted from the corpus. Each cell in the matrix contains a value indicating the presence or frequency of a term in a document. This structured representation of the text data allows us to apply machine learning techniques to uncover hidden patterns.\\

In the topic modeling phase, we use Latent Dirichlet Allocation (LDA), a popular machine learning technique for identifying topics within a set of documents. By applying LDA to the DTM, we transform the matrix into a space of topics. Specifically, LDA provides us with two key matrices: the Topic-Term Matrix ($\beta$) and the Document-Topic Matrix ($\theta$). The Topic-Term Matrix ($\beta$) indicates the probability that a term is associated with a specific topic, while the Document-Topic Matrix ($\theta$) indicates the probability that a document belongs to a specific topic.

To enhance this approach, we integrate predefined topic representations and refine document-topic assignments. In addition to extracting topics purely from the document-term matrix (DTM), we incorporate a set of predefined topics, namely [Fundamentals, Evaluation of Solutions, Trends], which are represented using a predefined set of keywords generated by a language generation model. This ensures that the model captures both the inherent structure of the dataset and domain-relevant themes. The predefined topics are processed through a keyword extraction function, which selects the most relevant words associated with each topic based on the provided corpus.

To construct a more robust topic representation, we create predefined topic matrices that integrate predefined topic vectors with the most relevant keywords extracted for each topic. These matrices are then normalized and combined with the LDA-generated topic distribution, ensuring a refined alignment of document-topic assignments. By doing so, we mitigate the limitations of purely unsupervised topic modeling, which might generate topics that lack semantic clarity.

Finally, to improve interpretability, we assign meaningful names to the discovered topics by combining the highest-probability terms from the topic-term matrix ($\beta$). Using a language generation model, we generate concise and descriptive topic names, ensuring that each topic is easily understandable. This process results in a refined set of topics ($T$), the most relevant terms ($K$), and a topic-term graph ($G_{tk}$) that illustrates relationships between topics and key terms. This methodology enhances the effectiveness of topic modeling by integrating both machine learning and domain-specific knowledge, leading to a more structured and meaningful representation of the analyzed corpus.

The final phase involves the generation and integration of knowledge representations, which include summaries, keywords, and interactive visualizations. For each topic $t$ in $T$, we identify the $N$ most relevant documents—those with the highest probabilities in $\theta$. This set, $D_t$, represents the documents most closely related to each topic. We then generate summaries of these documents, each with a maximum of $W_b$ words, using a language generation model. These summaries include references to the most relevant documents, which are added to the set $R_d$ if they are not already included. We also integrate all the summaries and generate $J$ suggested keywords using a language generation model. Additionally, we create interactive graphs from the $G_{tk}$ graph, showcasing nodes and relationships between the topics and terms, and highlighting significant connections.We also integrate all the summaries and generate $J$ suggested keywords using a language generation model. Additionally, the topic titles were improved based on the generated summaries using a language generation model, ensuring that the final topic names more accurately reflect the summarized content.
\\

This methodology provides a clear, structured approach to analyzing scientific literature, leveraging advanced NLP and machine learning techniques to generate useful and comprehensible knowledge representations. This is the methodology used in the development of this study, which presents the fundamentals, solution evaluation techniques, trends, and other topics of interest within this field of study. This structured approach ensures a thorough review and synthesis of the current state of knowledge, providing valuable insights and a solid foundation for future research.

\begin{table*}[!h]
	\caption{\centering Description of parameters of the proposed methodology}
	\resizebox{\textwidth}{!}{ % Ajusta el ancho de la tabla al texto
		\begin{tabular}{|l|l|}
			\hline
			\textbf{Parameter} & \textbf{Description}                                                            \\ \hline
			$T$                  & Set of topics to be generated with the LDA model. \\ \hline
			$\#T$                & Cardinality of $T$. That is, the number of topics (dimensions).                                    \\ \hline
			$K$                  & Set of common terms with the highest probability in topic $t$ used to label that topic. \\ \hline
			$\#K$ 				 & Cardinality of $K$.                                         \\ \hline
			$W_{t}$   			 & Maximum number of words for combining the common terms of topic $t$ into a new topic name.\\ \hline
			$N$                  & Number of the most relevant documents to generate a summary of topic $t$.\\ \hline
			$W_b$                & Maximum number of words to generate a summary of the list of $N$ most related documents to topic $t$.\\ \hline
			$J$                  & Number of suggested keywords to be generated as knowledge representation.\\ \hline
			$W_k$ 				 & Number of keywords selected by language generation model from the LDA-generated words to construct the predefined matrix. \\ \hline
		\end{tabular}
	}
	\label{tabDescription}
\end{table*}


\begin{figure*}[!h]
	\centering
	\includegraphics[width=1.0\textwidth]{Figures/method.png}
	\caption{Methodology for the generation of knowledge representations}
	\label{fig:Methodology}
\end{figure*}

%Actual
\begin{figure*}[!h]
	\centering
	\resizebox{\textwidth}{!}{ % Ajusta el ancho de la tabla a \textwidth
		\begin{tabular}{l}
			\hline
			\textbf{General Algorithm} \\
			\hline
			\textbf{Input:} Field of study, Scientific articles collected from virtual libraries, $\#T$, $\#K$, $W_{t}$, $N$, $W_b$, $J$, $W_k$\\
			\hline
			\textbf{Phase 1: Data Preparation with Natural Language Processing (NLP)} \\
			\quad a. Extract of text from documents. \\
			\quad b. Remove non-alphabetic characters that do not add value to the analysis. \\
			\quad c. Convert to lowercase and remove stopwords. \\
			\quad d. Apply lemmatization to transform words to their base form. \\
			\quad e. In each document, tokenize and identify n-grams to identify common terms (words or n-grams). \\
			\quad f. Unified generation of common terms of all documents. Where $TE$ is the unified set of common terms.\\
			\quad g. Add in $TE$ the basic terms for any field of study, such as: [Fundamentals, Evaluation of Solutions, Trends]. \\
			\quad h. Vectorize each document with respect to $TE$ and generate Document-Term Matrix (DTM).\\
			\quad \textbf{Output:} For each document [title, original text, common terms], and Document-Term Matrix (DTM)\\
			\textbf{Phase 2: Topic Modeling whit Machine Learning} \\
			\quad \textbf{Input:} Document-Term Matrix (DTM) \\
			\quad a. Apply Latent Dirichlet Allocation (LDA) to transform the DTM Matrix into a space of $\#T$ topics (dimensions).\\
			\quad b. Obtain the Topic-Term Matrix ($\beta$) that indicates the probability that a term is generated by a specific topic.\\ 
			\quad c. Obtain the Document-Topic Matrix ($\theta$) that indicates the probability that a document belongs to a specific topic.\\
			\quad d. Generate predefined topic representations using language generation model by sending the LDA-generated words and selecting $W_k$ keywords for each predefined topic.\\
			\quad e. Construct predefined matrices by combining predefined topic vectors with the $W_k$ selected keywords.\\
			\quad f. Normalize and combine the predefined matrices with the LDA-generated topic distribution to refine document-topic assignments, resulting in the new Matrix ($\beta$).\\
			\quad g. For each unknown $t$ topic in $\beta$, assign a name or label to the $t$ topic by combining the $\#K$ highest probability\\
			\quad \quad common terms in $\beta$ associated with that $t$ topic. This generates: \\
			\quad \quad The $T$ Set with the $\#T$ most relevant topics. \\
			\quad \quad The $K$ Set with the $\#K$ most relevant terms. \\
			\quad \quad The $G_{tk}$ Graph of the relationships between the most relevant topics ($T$) and the most relevant terms ($K$). \\
			\quad h. For each topic $t$ in $T$, modify topic $t$ by combining the common terms of $t$ into a new topic name with at most \\
			\quad \quad $W_{t}$ words using a language generation model.\\
			\quad \textbf{Output:} Relevant Topics ($T$), Topic-Term Matrix ($\beta$), Document-Topic Matrix ($\theta$) \\
			\textbf{Phase 3: Generation and Integration of Knowledge Representations} \\
			\quad \textbf{Input:} Relevant Topics $T$, Document-Topic Matrix ($\theta$) \\
			\quad \textbf{3.1: Knowledge representations through summaries and keywords}\\
			\quad \quad a. For each $t$ topic in $T$, obtain its $N$ most relevant documents, i.e., those with the highest probabilities in $\theta$. \\
			\quad \quad \quad Thus, $D_t$ represents the set of documents most related to each topic $t$.\\
			\quad \quad b. For each topic $t$ in $T$, and from $D_t$ generate a summary of the list of documents most related to that topic $t$ with \\
			\quad \quad \quad at most $W_b$ words using a language generation model. \\
			\quad \quad c. Incorporate into the summary the text citing references to the most relevant documents. Add these references to the \\ 
			\quad \quad \quad set $R_d$, which will contain all cited references, including new references if they have not been previously included.\\
			\quad \quad d. Integrate all the summaries and from them generate $J$ suggested keywords using a language generation model.\\
			\quad \quad e. Improve topic titles with a language generation model.\\
			\quad \textbf{3.2: Knowledge representations through knowledge visualizations with interactive graphs}\\
			\quad \quad a. From the $G_{tk}$ graph, generate an interactive graph with nodes and relationships between the topics $T$\\
			\quad \quad and the $K$ most relevant terms.\\
			\quad \quad b. Highlight the connections between the topics $T$ and the $K$ most relevant terms.\\
			\quad \textbf{3.3: Integration of Knowledge Representations}\\
			\hline
			\textbf{Output:} Knowledge Representations \\
			\hline
		\end{tabular}
	}
	\caption{\centering General algorithm of the methodology incorporating natural language processing, machine learning techniques and language generation models}
	\label{tab:Algorithm}
\end{figure*}

\FloatBarrier


\section{Fundamental of brain cancer deep learning}
The utilization of deep learning techniques for the detection and classification of brain cancer has garnered significant attention in recent years, particularly with the advent of microwave imaging technologies. Microwave brain imaging (MBI) systems represent a novel approach for early tumor detection, capitalizing on the unique electromagnetic properties of tissues. The fine-tuned feature-extracted deep transfer learning model (FT-FEDTL) is a notable advancement in this domain. It employs the InceptionV3 architecture for feature extraction, enhancing the classification performance of brain tumor images through fine-tuning of additional layers, which optimizes hyperparameters effectively.

The primary challenge in brain cancer diagnosis lies in the intricate patterns and shapes of tumors, which complicate manual identification and classification by radiologists. The FT-FEDTL model addresses these challenges by automating the classification process, thus improving efficiency and accuracy. Utilizing a dataset of 4200 microwave-based images, the model categorizes tumors into six distinct classes: normal tissue, single benign tumor, single malignant tumor, two benign tumors, two malignant tumors, and a combination of benign and malignant tumors. The model demonstrated remarkable performance, achieving an overall accuracy of 99.65	extbackslash{}%, along with high recall, precision, specificity, and F-score values, indicating its potential utility in clinical settings to support radiologists in tumor classification tasks.

Moreover, the inclusion of data augmentation techniques helped mitigate issues of class imbalance, thereby enhancing the robustness of the model. In comparative evaluations, the FT-FEDTL outperformed traditional machine learning approaches and various pretrained models, establishing its superiority in classifying brain tumors from microwave images. The architecture's ability to extract features efficiently and adapt to specific datasets through transfer learning methods further contributes to its effectiveness.

The implications of this research extend beyond mere classification, as it highlights the critical role of deep learning in revolutionizing brain cancer detection methodologies. The automated nature of the FT-FEDTL model not only accelerates the diagnostic process but also potentially reduces the cognitive load on medical professionals, allowing them to focus on more complex cases. As a result, the model represents a significant step toward integrating advanced artificial intelligence solutions in the medical imaging domain, facilitating earlier and more accurate detection of brain tumors.

The study emphasizes the importance of continued research into deep learning frameworks and their applications in medical imaging, encouraging further exploration of diverse datasets and optimization techniques to enhance detection capabilities for various types of cancers.
\begin{figure}[h]
\centering
\includegraphics[width=1\textwidth]{Figures/lda_topic_graph_simplified.png}
\caption{Grafo de relaciones temáticas generado mediante LDA, donde los nodos representan temas identificados y sus palabras clave asociadas. La palabra clave central conecta los temas, y las palabras compartidas aparecen enlazadas a múltiples etiquetas según su relevancia en el modelo. Los pesos de los enlaces reflejan la importancia de cada término dentro de su tópico.}
\end{figure}
\section{Evaluation of brain cancer deep learning}
Deep learning (DL) techniques have increasingly been applied to the study of glioblastoma (GBM) and other brain tumors, particularly in the analysis of histopathological images. This systematic review highlights the evolving role of machine learning (ML) and DL methodologies in enhancing our understanding of GBM pathology through histological data. Despite their potential, ML/DL applications in GBM histopathology have been relatively limited, with only a small number of studies integrating bioinformatics with histopathological analysis, particularly in the context of GBM.

The review identifies a total of 54 eligible studies, revealing a clear trend towards utilizing ML/DL techniques to improve diagnostic accuracy and prognostic evaluations for GBM. Notably, only eight studies have effectively correlated histopathological data with -omics data, reflecting a significant gap in current research. The most frequently employed models included Support Vector Machines (SVM) and convolutional neural networks (CNNs) based on ResNet architecture, which were utilized to gain insights into GBM pathogenesis by contextualizing histological data with corresponding molecular markers.

GBM is characterized by its heterogeneity, which complicates treatment and prognosis. The review also notes the consensus classification between IDH-wildtype and IDH-mutant GBM, emphasizing the importance of identifying distinct molecular signatures within tumor subtypes. Recent advancements have shown that ML/DL techniques can effectively extract complex patterns from histopathological images, enabling more personalized treatment approaches and improved patient outcomes.

However, the review underscores several limitations within the existing studies, particularly regarding the clarity of reported methodologies for training and evaluating models. Many studies did not adequately describe their training datasets or how they were split into training and testing sets, which poses challenges to replicability and generalizability. Furthermore, a notable trend was observed wherein a significant number of studies relied on single-source datasets, potentially limiting the robustness of their findings.

The systematic review concludes that while there is a growing interest in ML/DL applications within GBM histopathological research, further efforts are needed to integrate multi-omics data with histological analysis to enhance model performance and provide deeper biological insights. The authors advocate for standardized reporting practices concerning data sourcing and model evaluation metrics to facilitate collaboration and the development of more comprehensive diagnostic tools in the clinical setting.

In summary, the integration of ML/DL techniques in GBM research is poised to significantly advance the field, provided that researchers address the current methodological shortcomings and continue to explore novel avenues for analysis and integration of data types.
\section{Brain tumor classification}
Automated systems for brain cancer diagnosis have become increasingly vital due to the complexity and variability of brain tumors, which are often characterized by their heterogeneous nature and low survival rates. The classification of brain tumors, particularly through the analysis of MRI images, presents significant challenges that can benefit from advancements in deep learning techniques. In recent years, deep learning, particularly through Convolutional Neural Networks (CNNs) and their variants like Residual Networks (ResNet), has demonstrated substantial improvements in image classification tasks, including those related to medical imaging 	\cite{Sarah_2020}.

The proposed model in the study employs an enhanced ResNet architecture that facilitates the classification of MRI images of three primary brain tumor types: meningiomas, gliomas, and pituitary tumors. The model is trained on a benchmark dataset consisting of 3064 annotated MRI images. By implementing various data augmentation techniques, the research aims to maximize the dataset's utility, thereby enhancing the model's performance and generalization capabilities. These techniques include transformations such as flipping, rotation, shifting, and brightness manipulation, which are critical in addressing the issues related to the limited size of medical datasets 	\cite{Sarah_2020}.

Evaluation metrics such as accuracy, precision, recall, F1 score, and balanced accuracy were employed to assess the model's performance. The results indicate that the model achieved a remarkable accuracy of 99	extbackslash{}% at the image-level and 97	extbackslash{}% at the patient-level, surpassing previous studies that reported lower accuracy rates. This achievement underscores the efficacy of the ResNet architecture in managing the complexities inherent in MRI scans of brain tumors, as it can learn and leverage high-level features more effectively than traditional machine learning approaches 	\cite{Sarah_2020}.

Moreover, the study emphasizes the importance of developing a Computer-Assisted Diagnosis (CAD) system that supports radiologists in diagnosing and classifying brain tumors. The integration of deep learning techniques into CAD systems can significantly alleviate the workload on medical professionals, enhance diagnostic accuracy, and ultimately improve patient outcomes. It addresses the limitations of manual diagnosis, which can be error-prone and reliant on the radiologist's experience 	\cite{Sarah_2020}.

Despite its promising results, the study acknowledges the persistent challenges of classifying brain tumors due to the overlapping characteristics of different tumor types and the variability in MRI image quality. Future research directions include experimenting with larger datasets and exploring additional tumor types to further validate and enhance the proposed model's applicability 	\cite{Sarah_2020}. The integration of deep learning into the diagnostic process represents a significant advancement in the field of medical imaging, suggesting a paradigm shift towards more automated and accurate diagnostic tools for brain cancer classification.
\section{Slca unet segmentation}
The segmentation of brain tumors is a critical task in medical imaging, particularly using magnetic resonance imaging (MRI). Traditional methods such as manual segmentation are time-consuming and susceptible to variability due to the complex tumor shapes and boundaries. Recent advancements in deep learning, especially through the utilization of convolutional neural networks (CNNs) and architectures like UNet, have shown promise in automating and enhancing the accuracy of brain tumor segmentation. However, traditional UNet models still face challenges related to feature extraction and computational efficiency 	\cite{P.S._2025}.

The study introduces a novel architecture, the Stacked Convolution Layered Channel Attention UNet (SLCA-UNet), designed to tackle these issues. This architecture integrates residual dense blocks and various attention mechanisms to improve the segmentation process by capturing both fine and coarse features more effectively. By incorporating layered attention, the SLCA-UNet enhances the model's contextual understanding, allowing it to focus on relevant tumor characteristics while minimizing information loss during feature extraction 	\cite{P.S._2025}. 

Experimental results demonstrate that the SLCA-UNet significantly outperforms traditional methods and other contemporary models in terms of accuracy metrics, achieving a Dice Similarity Coefficient (DSC) of 0.921 for whole tumors, and improving segmentation precision for core and enhancing tumor cases as well 	\cite{P.S._2025}. This advancement is crucial as it leads to quicker and more reliable diagnostic outcomes, which are essential for timely treatment interventions.

Additionally, the architecture has been validated on several benchmark datasets, including the BraTS 2020 dataset, showcasing its robustness and adaptability across different types of brain tumors and MRI modalities. The SLCA-UNet's ability to integrate various imaging features allows for improved segmentation accuracy and efficiency, which can enhance clinical workflows and patient outcomes 	\cite{P.S._2025}.

Future directions for this research will include exploring the application of SLCA-UNet in real-world clinical settings, extending its capabilities to other imaging modalities, and optimizing the model for deployment in resource-constrained environments. There is also potential for further enhancing the model through the incorporation of multimodal data and advanced loss functions to address challenges such as class imbalance in tumor segmentation 	\cite{P.S._2025}.

In summary, the proposed SLCA-UNet architecture represents a significant step forward in the field of brain tumor segmentation, leveraging deep learning to automate and improve the accuracy of this critical diagnostic task.
\section{Deep learning glioblastoma}
The application of deep learning (DL) in brain cancer analysis, particularly glioblastoma (GBM), represents a significant advancement in the field of histopathology. Glioblastoma is characterized by its high degree of malignancy and heterogeneity, making its diagnosis and treatment particularly challenging. Recent studies have shown that DL models can effectively analyze histopathological images, providing insights into GBM's complex pathogenesis. A systematic review of the literature reveals an emerging trend in leveraging DL to uncover relationships between histopathological data and molecular characteristics of GBM, which could lead to more personalized treatment approaches.

Machine learning (ML) and DL techniques have been utilized to enhance the segmentation and classification of brain tumors from histopathological images. Specifically, convolutional neural networks (CNNs) are commonly employed for these tasks, demonstrating superior performance in recognizing intricate patterns that are often beyond human capability. For instance, studies have indicated that CNN architectures, including ResNet and U-Net, are particularly effective in segmenting tumor regions and classifying tumor grades based on histological features. This ability to automate and improve diagnostic accuracy not only aids pathologists but also supports better clinical decision-making.

The review of 54 studies highlights that a significant number focused on GBM-related research, showcasing a variety of methodologies and data types utilized in conjunction with histopathological images. These studies predominantly relied on hematoxylin and eosin (H	extbackslash{}\&E) stained images, often integrating additional data such as genomic and proteomic profiles to provide a more comprehensive understanding of tumor behavior and patient outcomes. Notably, while many studies achieved high accuracy in tumor classification, a recurring limitation was the insufficient reporting of model training and evaluation methodologies, which raises concerns regarding the reproducibility of results.

Moreover, the integration of bioinformatics with histopathological analysis has emerged as a key area for future exploration. This approach allows for the contextualization of histological data with molecular insights, offering potential pathways for uncovering tumor heterogeneity and therapeutic vulnerabilities. For example, studies have successfully correlated histopathological features with gene expression profiles, enhancing predictive modeling for patient survival and treatment responses.

Despite the promising developments, the literature indicates that many studies still prioritize model performance metrics over comprehensive biological insights. This trend underscores a critical need for future research to adopt a more integrative approach in analyzing GBM. By combining histopathological data with robust ML/DL frameworks, researchers can better address the complexities of tumor biology and improve diagnostic and prognostic capabilities.

In conclusion, the increasing application of DL in brain cancer histopathology, particularly in GBM, reflects a transformative shift towards more data-driven, personalized medicine. However, to fully harness the potential of these technologies, researchers must prioritize methodological transparency and the integration of diverse biological data, ultimately leading to improved outcomes for patients diagnosed with this aggressive form of cancer.

.
\section{Trends of brain cancer deep learning}
Recent advancements in deep learning have significantly impacted brain cancer detection, particularly through the development of innovative models like the Fine-tuned Feature Extracted Deep Transfer Learning (FT-FEDTL) model. This model leverages microwave-based imaging to classify brain tumors into multiple categories, thus improving diagnostic accuracy and efficiency. The FT-FEDTL approach utilizes the InceptionV3 architecture as its backbone, incorporating fine-tuned layers to enhance classification performance while automatically identifying and categorizing brain tumor images.

The microwave brain imaging system is emerging as a promising diagnostic tool, offering advantages over conventional imaging modalities such as MRI and CT scans, including non-ionizing radiation and cost-effectiveness. However, manual tumor identification remains a challenge due to the complex shapes and patterns of tumors. The FT-FEDTL model addresses these challenges by implementing a deep transfer learning technique that allows for effective classification of brain tumors, achieving impressive accuracy rates of up to 99.65	extbackslash{}%, along with high precision and recall metrics.

The study emphasizes the importance of early detection, which is crucial for improving patient outcomes. With brain cancer incidence rates projected to increase, the need for accurate and efficient diagnostic methods like the FT-FEDTL model is evident. This model not only enhances the capability of radiologists in diagnosing tumors but also supports timely medical interventions, ultimately leading to better survival rates.

Comparative analysis with traditional methods reveals that the deep learning-based approaches, particularly those utilizing transfer learning, outperform conventional machine learning techniques. The ability to handle imbalanced datasets and the automatic feature extraction capabilities of deep learning models contribute to their superior performance in tumor classification. The integration of data augmentation techniques further strengthens the robustness of these models, making them more adept at generalizing across different imaging scenarios.

Despite these advancements, there are still challenges that need to be addressed, such as the interpretability of deep learning models and the necessity of high-quality annotated datasets. Future research directions include enhancing model transparency, improving dataset diversity, and exploring hybrid approaches that combine the strengths of different algorithms. Overall, deep learning presents a transformative approach to brain cancer detection, paving the way for more effective diagnosis and treatment in clinical practice.
\section{Related works and original contributions of the paper}
The systematic review conducted by Vong et al. (2025) provides a comprehensive overview of the application of machine learning (ML) and deep learning (DL) techniques in the field of glioblastoma (GBM) histopathology. It underscores the evolution of these methodologies and highlights the insufficient integration of bioinformatics with histopathological analysis, emphasizing the need for better reporting of methodologies to enhance the robustness and reproducibility of findings. The study found a growing trend towards using ML/DL approaches to reveal relationships between biological data and GBM pathology, yet observed that only eight of the 54 studies reviewed effectively utilized -omics data. This work is particularly valuable as it sets a foundation for future research to address the methodological gaps noted and promotes the integration of various biological data types to advance GBM understanding and treatment options \cite{Chun_2025}.

Ahmad and Alqurashi's (2024) survey explores the critical role of deep learning in early cancer detection through medical imaging, encompassing a wide range of cancer types including brain cancer. Their analysis of 99 articles reveals the limitations of manual interpretation of imaging data, advocating for automated techniques that enhance diagnostic accuracy and efficiency. The review covers various aspects of deep learning applications, from image preprocessing to feature extraction and evaluation metrics. By summarizing diverse methodologies and future challenges, this study provides significant insights into the potential pathways for improving cancer detection through advanced imaging techniques, highlighting the importance of continuing research in this fast-evolving landscape \cite{Istiak_2024}.

The contributions of both studies are integral to the ongoing discourse on improving cancer diagnosis through advanced machine learning techniques. Vong et al. focus specifically on the application of ML/DL in analyzing the histopathological aspects of glioblastoma, while Ahmad and Alqurashi provide a broader perspective on the role of medical imaging across multiple cancer types. Together, they show how the integration of complex datasets, enhanced methodologies, and innovative deep learning models can lead to improved diagnostic accuracy and more personalized treatment strategies. The importance of robust methodological frameworks and the potential for leveraging multimodal data in cancer research are common threads that emphasize the transformative impact of deep learning on the medical field. This combined perspective offers a richer understanding of how these technologies can contribute to earlier detection and better patient outcomes in the context of brain tumors and beyond.

\begin{itemize}[label=\textbullet]
    \item User a machine learning-based methodology to select the most relevant papers, standing out the latest and most cited papers in the area.
    \item Provides a comprehensive survey on deep learning advancements in brain cancer detection, highlighting novel models, their accuracy, and implications for clinical diagnosis and treatment.
    \item Fundamentals of Brain Cancer Deep Learning demonstrates how deep learning techniques enhance brain cancer detection through microwave imaging, improving classification accuracy and efficiency for clinical applications.
    \item Evaluation of Brain Cancer Deep Learning highlights the potential of machine learning to improve glioblastoma diagnostics, emphasizing the need for integrating multi-omics data with histopathological analysis for better outcomes.
    \item Brain Tumor Classification showcases an advanced deep learning model that significantly improves MRI image classification accuracy, aiding radiologists in diagnosing and classifying various brain tumor types.
    \item SLCA UNet Segmentation presents a novel architecture that enhances brain tumor segmentation accuracy in MRI, addressing traditional method limitations and improving diagnostic workflows in clinical settings.
    \item Deep Learning Glioblastoma reveals the effectiveness of deep learning in analyzing glioblastoma pathology, emphasizing the integration of histopathological and molecular data for personalized treatment strategies.
    \item Trends of Brain Cancer Deep Learning illustrates the evolution of deep learning models in brain cancer detection, highlighting their superior diagnostic capabilities compared to conventional methods and emphasizing early detection's importance.
\end{itemize}
\section{Conclusions}
The utilization of deep learning techniques in brain cancer diagnosis has emerged as a critical advancement in medical imaging, particularly through the development of models like the Fine	extendash{}tuned Feature Extracted Deep Transfer Learning (FT	extendash{}FEDTL) model. This research highlights the pressing need for effective diagnostic tools as brain cancer incidence rates continue to rise. Key findings include the model's impressive classification accuracy of 99.65% when analyzing microwave imaging data, its ability to automate the classification of tumors into distinct categories, and its superior performance over traditional methods, addressing the challenges posed by tumor complexity and shape variability. Additionally, the study underscores the importance of integrating data augmentation techniques to enhance model robustness and mitigate class imbalance. Despite these advancements, limitations persist, such as the need for high	extendash{}quality annotated datasets and improved model interpretability. Future research should focus on enhancing methodological transparency, exploring multimodal data integration, and optimizing deep learning models for clinical application. This approach will not only facilitate earlier and more accurate tumor detection but also support radiologists in their diagnostic efforts, ultimately leading to better patient outcomes. The findings signify a transformative shift towards more automated and precise diagnostic tools in the realm of brain cancer detection, advocating for continued exploration of innovative methodologies in this rapidly evolving field.
\begin{thebibliography}{00}
    \bibitem{P.S._2025} P.S. Tejashwini and J. Thriveni and K.R. Venugopal (2025). A novel SLCA-UNet architecture for automatic MRI brain tumor segmentation.
\bibitem{Chun_2025} Chun Kiet Vong and Alan Wang and Mike Dragunow and Thomas I-H. Park and Vickie Shim (2025). Brain tumour histopathology through the lens of deep learning: A systematic review.
\bibitem{Sarah_2020} Sarah Ali {Abdelaziz Ismael} and Ammar Mohammed and Hesham Hefny (2020). An enhanced deep learning approach for brain cancer MRI images classification using residual networks.
\bibitem{Istiak_2024} Istiak Ahmad and Fahad Alqurashi (2024). Early cancer detection using deep learning and medical imaging: A survey.
\bibitem{Amran_2024} Amran Hossain and Rafiqul Islam and Mohammad Tariqul Islam and Phumin Kirawanich and Mohamed S. Soliman (2024). FT-FEDTL: A fine-tuned feature-extracted deep transfer learning model for multi-class microwave-based brain tumor classification.
\bibitem{b1}
    Hurtado, Remigio; Picón, Cristian; Muñoz, Arantxa; Hurtado, Juan.
    "Survey of Intent-Based Networks and a Methodology Based on Machine Learning and Natural Language Processing."
    In Proceedings of Eighth International Congress on Information and Communication Technology.
    Springer Nature Singapore, Singapore, 2024.
    \bibitem{b2}
    Park, Keunheung; Kim, Jinmi; Lee, Jiwoong. 
    ``Visual Field Prediction using Recurrent Neural Network,'' 
    \emph{Scientific Reports}, 
    vol. 9, no. 1, p. 8385, 
    2019, 
    https://doi.org/10.1038/s41598-019-44852-6.
    
    \bibitem{b3}
    Xu, M.; Du, J.; Guan, Z.; Xue, Z.; Kou, F.; Shi, L.; Xu, X.; Li, A. 
    ``A Multi-RNN Research Topic Prediction Model Based on Spatial Attention and Semantic Consistency-Based Scientific Influence Modeling,'' 
    \emph{Comput Intell Neurosci}, 
    vol. 2021, 
    2021, 
    p. 1766743, 
    doi: 10.1155/2021/1766743.
    
    \bibitem{b4}
    Kreutz, Christin; Schenkel, Ralf.
    ``Scientific Paper Recommendation Systems: a Literature Review of recent Publications,''
    2022/01/03.
	\bibitem{Hurtado2023} Hurtado, R., et al. "Survey of Intent-Based Networks and a Methodology Based on Machine Learning and Natural Language Processing." International Congress on Information and Communication Technology. Singapore: Springer Nature Singapore, 2023.
    \bibitem{Lopez2024} Lopez, A., Dutan, D., Hurtado, R. "A New Method for Predicting the Importance of Scientific Articles on Topics of Interest Using Natural Language Processing and Recurrent Neural Networks." In: Yang, X.S., Sherratt, S., Dey, N., Joshi, A. (eds) Proceedings of Ninth International Congress on Information and Communication Technology. ICICT 2024 2024. Lecture Notes in Networks and Systems, vol 1013. Springer, Singapore. https://doi.org/10.1007/978-981-97-3559-4\_50.
\end{thebibliography}
\end{document}